\documentclass[a4paper,12pt]{article}
\usepackage{graphicx}
\usepackage{float}
\usepackage{color}

\begin{document}
\title{Assignment 5}
\author{Emilio Becerra}
\date{\ March 15, 2024}
\maketitle

\section{Estimation}
Using the estimation sample, estimate a binary logit model to model the
probability of a customer buying from the catalog using all variables available to the
consumer. Report and explain whether the estimates are reasonable or not.
\vspace{12pt}
\begin{table}[H] \centering
\begin{tabular}{ll}
Variable Names \\
\hline
Variable & Description\\
\hline
customer no & Customer id, can be linked to address \\
buytabw & 1 = bought, 0 = did not buy from catalog \\
tabordrs & Total orders from tabloids \\
divsords & Orders with shoe division \\
divword & Orders with women’s division \\
spgtabord & Total spring tabloid orders \\ 
moslsdvs & Months since last shoe order \\
moslsdvw & Months since last women’s order \\
moslstab & Months since last tab order \\ 
orders  & Total orders \\
holdout & = 1 if part of the holdout sample \\
\hline
\end{tabular}
\end{table}
\input{regression_table.tex}
\vspace{12pt}
\textbf{ANS:} 
Every variable except divords: ``Orders with shoe division'' is significant.
All of the ``months since...'' variables have a negative magnitude which makes sense:
We expect that the longer consumers go without buying an item, the less likely they 
are in buying from the catalog when it gets mailed to them. The rest of the variables
are positive and we expect them to be. They all specifiy customer sales. The more items
each customer has bought the more likely we expect them to buy from the catalog.

\section{Validation}
Now switch to the holdout worksheet (holdout = 1) and perform your
analysis there. Using the coefficients from the previous estimation produces probability
estimates for a validation model. In other words, the previous model estimates will be
used to make predictions in the holdout sample.
\begin{itemize}
\item Explain how we should interpret this event probability and why we performed
this step using the holdout sample.
\end{itemize}
\textbf{ANS:} 
Using the holdout sample we can do two things:\\ 
1. We can find our prediction for each customer being a 0 or 1 (buy or no buy)\\
2. We can find the probability of them buying from the catalog ( = 1).\\
But in order to properly use the holdout sample it was essential we run our model 
using the estimation sample, which was derived from the raw catalog.csv dataset. This 
raw data represents the company's historical consumer data. What we inevitably want 
to do is make a valid prediction of future consumer behavior. Since we don't have the
``real world'' future consumer data for `buytabw', we must use the data that we 
already have to make a reliable model. What we do is split the raw data in half. 
The estimation sample group serves to create a machine/model to bring over to the 
holdout sample group. We ran the regression and found that the overall model was 
significant. In the holdout sample group we obviously have the historical ``buytabw''
data, but we are going to pretend it doesn't exist to simulate the ``real world''. 
We predict the outcome for each consumer and get their respective probabilities. 
All that is left to do now is see how well the model did in estimating ``buytabw''.
We have the prediction and we have the actual purchase decision. We can look at several
graphs and tables to better interpret the accuracy.\\
Here is a confusion matrix with the accuracy results my model found:
\vspace{12pt}
\begin{table}[H]
\begin{tabular}{l|l|l}
Actual Decision/Prediction &Did Buy& Did Not Buy \\
\hline
Predicted Buy & 8355 & 193 \\
\hline
Predicted Did not Buy & 1475 & 291 \\
\end{tabular}
\end{table}
As we can see, the model wasn't perfect in predicting. We should be skeptical moving forward. 
\section{Test}
In the holdout sample, create box plots using the actual purchase decision on
the x-axis and the predicted purchase probability on the y-axis. Interpret the graph.
Do you think the estimated event probability separates those consumers who actually
bought in the holdout sample?
\begin{figure}[H]
\centering
\includegraphics[width=1\textwidth]{figure_1.png}
\label{Box Plots of Actual Purchase Decision and Predicted Purchase Probability}
\end{figure}
\textbf{ANS:} 
I would say that it attempts to separate, but strictly, it does not separate them.
As you can see, the box plots interlap with one another. You cannot truly say that we
are able to completely separate consumers into two different groups. The outliers
are also concerning. 

\section{Categorize consumers into groups}
Create “rank,” a variable that ranks the cus-
tomers according to their predicted purchase probability. Then, assign each consumer
to a group, creating a variable called group numbered 1 to 10, where 1 indicates the
10\% of customers with the lowest purchase probability, and 10 indicates the 10\% of cus-
tomers with the highest purchase probability. Notice we are separating the consumers
in deciles.
\begin{itemize}
\item Provide box plots with the group variable on the x-axis and the predicted purchase
probabilities on the y-axis
\end{itemize}
\begin{figure}[H]
\centering
\includegraphics[width=1\textwidth]{figure_2.png}
\label{Box Plots of Actual Purchase Decision and Predicted Purchase Probability}
\end{figure}
\textbf{ANS:} 
As group number approaches group 10, the deciles increase in probability of buying
from the catalog. 

\section{Analyze the marketing strategy}
At this point we realize that each group has a
different purchase probability; now we calculate how many buyers we capture by mail-
ing to each group. The main idea is to order the decile groups that we just created into
those who are most likely to buy (group 10) to those who are less likely to buy (group
1) and calculate how many buyers we capture per additional group. For example, you
will find out that although group 10 comprises 10\% of all the samples, it captures
about 30\% of all buyers. In contrast, group 1, also with 10\% of all observations, does
not have any buyers.
Create two new columns in your table created in part (5): 1) the cumulative number of
observations (e.g., no. of obs. for groups 10 to 1) and 2) the cumulative no of buyers.
Create a chart with the cumulative \% of mailed tabloids (10\% for each group) in the
horizontal axis and the \% of buyers captured by mailings to that group in the vertical
axis. This is referred in marketing as the ”Gains” chart. Does the gains chart indicate
that the response model can be useful to the catalog retailer?

\begin{figure}[H]
\centering
\includegraphics[width=1\textwidth]{figure_3.png}
\label{Gains Chart}
\end{figure}
\textbf{ANS:} 
The Gains Chart indicates a positive relationship. We notice that every dot indicates 
each decile group. The area under the curve going up to the first dot represents 
the \% of the first decile group. In group 10 (which on the x-axis is 90-100\%),
the area under the curve is the largest here. This means that when we mail 10\% of our
mailed tabloids, we capture a certain \% of buyers, which was said to be 30\%.

\section{Estimate Profits}
Based on past data, the average margin per customer is \$19.5.
The cost of printing and mailing one tabloid is \$1.0. Using the predicted purchase
probability, calculate expected profits (for each obs). Provide a histogram of the profit
variable. Interpret the graph. What fraction of customers is profitable in expectation?
\begin{figure}[H]
\centering
\includegraphics[width=1\textwidth]{figure_4.png}
\textbf{ANS:} 
The highest frquency are those customers who actually cause the company financial 
loss. Then the distribution of expected profit is skewed right, meaning that the majority
of the customers that give positive profit are going to give the company on average of
\$2. However, there are customers who in expectation give the company more than 
\$17.5 of profit.
I calculated that the fraction of customers that are profitable in expectation: 0.69

\label{Gains Chart}
\end{figure}
\section{Decision about the Optimal Mailing Strategy}
What mailing strategy do you recommend? Compare the actual profitability 
(you know if they bought or not) from this strategy to the expected profitability based on 
your model. I.e., for each customer who received a catalog, you compare expected profits 
and actual profits, given the buytabw variable.

Compare your mailing strategy to a blanket mailing strategy, where all customers
receive a catalog. What is the percent improvement in profits relative to the blanket
mailing strategy? \\

\textbf{ANS:}\\
average margin per customer = 19.5\\
\vspace{12pt}
cost of printing and mailing = 1.0\\
Actual profits from customers who bought: \$32671.00\\
\begin{itemize}
\item I found this by doing: \\
num buyers = (holdout sample['buytabw'] == 1).sum()\\
actual profits = num buyers * (average margin per customer - cost of printing and mailing)\\
\end{itemize}
Expected profits if we only mailed to profitable customers: \$132219.50\\
\begin{itemize}
\item I found this by doing:\\
total observations = len(holdout sample)\\
num profitable observations = int(total observations * fraction profitable)\\
\end{itemize}
Which gives me the number of profitable customers. I then multiply that buy the profit margin:\\
\begin{itemize}
\item expected profits profitable customers = num profitable observations *\\
(average margin per customer - cost of printing and mailing)\\
\end{itemize}
Expected profits if we mailed to every customer: \$24123.00\\
\begin{itemize}
\item I found this buy summing every customer's profit margins.\\
\end{itemize}
Percentage Change Improvement going from Blanket Mailing to Profitable Customer Only Mailing: 81.76\%

\end{document}

